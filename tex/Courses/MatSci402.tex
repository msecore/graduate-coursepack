%404-------------------------------------------------------
	\subsection{MAT{\_}SCI 402: Structure of Crystalline and Non-crystalline Materials}
	\vspace{-0.5em} \hfill \rule{0.4\textwidth}{.4pt}\newline
	\null \hfill \NUBrandTwoLight{\large{James Rondinelli}} \newline
	\null \hfill \NUBrandTwoLight{\large{\textit{Fall Term}}}
%-----------
\normalfont
	\subsubsection*{Course Description}
	This course centers on the methods of describing both crystalline and non-crystalline materials, as well as addresses the underlying relationships of materials and their structures. Symmetry principles and tensor representations of crystal properties will be used to describe, for example, transport behavior, elasticity, displacement phase transitions, and material anisotropy. Topics in non-crystalline solids include theories and models of amorphousness, as well as influence of ordering in hard and soft condensed matter systems such as glasses and polymers.
	\begin{displayquote}
			\vspace{-0.5em}
			\subsubsection*{\normalfont{\color{black} 1. Symmetry Principles, Point, Plane, and Space Groups}} \label{Topic402-1}
				\addcontentsline{toc}{subsubsection}{\nameref{Topic402-1}}
				\vspace{-0.5em}
			%---------
			\subsubsection*{\normalfont{\color{black} 2. Crystallographic Descriptors and Structure}} \label{Topic402-2}
				\addcontentsline{toc}{subsubsection}{\nameref{Topic402-2}}
				\vspace{-0.5em}
			%---------			
			\subsubsection*{\normalfont{\color{black} 3. Material Anisotropy}} \label{Topic402-3}
				\addcontentsline{toc}{subsubsection}{\nameref{Topic402-3}}
				\vspace{-0.5em}
			%---------
				\subsubsection*{\normalfont{\color{black} 4. Displacive Phase Transitions}} \label{Topic402-4}
				\addcontentsline{toc}{subsubsection}{\nameref{Topic402-4}}
				\vspace{-0.5em}
			%---------
				\subsubsection*{\normalfont{\color{black} 5. Tensors and Constitutive Relations}} \label{Topic402-5}
				\addcontentsline{toc}{subsubsection}{\nameref{Topic402-5}}
				\vspace{-0.5em}
			%---------				
				\subsubsection*{\normalfont{\color{black} 6. Sphere Packing and Non-crystalline Solids}} \label{Topic402-6}
				\addcontentsline{toc}{subsubsection}{\nameref{Topic402-6}}
				\vspace{-0.5em}
			%---------				
			\subsubsection*{\normalfont{\color{black} 7. Polymer Chemistry and Architecture}} \label{Topic402-7}
				\addcontentsline{toc}{subsubsection}{\nameref{Topic402-7}}
				\vspace{-0.5em}
			%---------
			\subsubsection*{\normalfont{\color{black} 8. Molecular Order in Soft Condensed Matter}} 	\label{Topic402-8}
				\addcontentsline{toc}{subsubsection}{\nameref{Topic402-8}}
		\end{displayquote}

%-----------
\subsubsection*{Main Texts:} 
		\begin{itemize}
			\item \bibentry{nye1985physical}
			\item \bibentry{de2012structure}
			\item \bibentry{jones2002soft}
		\end{itemize}
		%
%\subsubsection*{Supplementary Texts:} 
			%\begin{itemize}
				%\item Point Defects ``Diffusion in Solids,'' Paul G. Shewmon (TMS, Pittsburgh, PA) 2\textsuperscript{nd} Edition 
				%\item ``Diffusion in Solids: Field Theory, Solid-State Principles and Applications,'' Martin E. Glicksman (John Wiley \& Sons, Inc., New York, 2000) 
				%\item ``Elementary Dislocation Theory,'' Johannes Weertman and Julia R. Weertman (Oxford University Press, 1992), 2\textsuperscript{nd} Edition Dislocations \& Planar Imperfections 
			%\item ``Theory of Dislocations,'' John Price Hirth and Jens Lothe (John Wiley \& Sons, New York, 1982), 2\textsuperscript{nd} Edition
			%\end{itemize}

%-----------
%\subsubsection*{Instructor Suggestions:}
%Not yet received.